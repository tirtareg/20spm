% Options for packages loaded elsewhere
\PassOptionsToPackage{unicode}{hyperref}
\PassOptionsToPackage{hyphens}{url}
%
\documentclass[
]{book}
\usepackage{lmodern}
\usepackage{amssymb,amsmath}
\usepackage{ifxetex,ifluatex}
\ifnum 0\ifxetex 1\fi\ifluatex 1\fi=0 % if pdftex
  \usepackage[T1]{fontenc}
  \usepackage[utf8]{inputenc}
  \usepackage{textcomp} % provide euro and other symbols
\else % if luatex or xetex
  \usepackage{unicode-math}
  \defaultfontfeatures{Scale=MatchLowercase}
  \defaultfontfeatures[\rmfamily]{Ligatures=TeX,Scale=1}
\fi
% Use upquote if available, for straight quotes in verbatim environments
\IfFileExists{upquote.sty}{\usepackage{upquote}}{}
\IfFileExists{microtype.sty}{% use microtype if available
  \usepackage[]{microtype}
  \UseMicrotypeSet[protrusion]{basicmath} % disable protrusion for tt fonts
}{}
\makeatletter
\@ifundefined{KOMAClassName}{% if non-KOMA class
  \IfFileExists{parskip.sty}{%
    \usepackage{parskip}
  }{% else
    \setlength{\parindent}{0pt}
    \setlength{\parskip}{6pt plus 2pt minus 1pt}}
}{% if KOMA class
  \KOMAoptions{parskip=half}}
\makeatother
\usepackage{xcolor}
\IfFileExists{xurl.sty}{\usepackage{xurl}}{} % add URL line breaks if available
\IfFileExists{bookmark.sty}{\usepackage{bookmark}}{\usepackage{hyperref}}
\hypersetup{
  pdftitle={Standar Pelayanan Minimal},
  pdfauthor={Registrasi Mahasiswa Universitas Sultan Ageng Tirtayasa Tahun 2018},
  hidelinks,
  pdfcreator={LaTeX via pandoc}}
\urlstyle{same} % disable monospaced font for URLs
\usepackage{longtable,booktabs}
% Correct order of tables after \paragraph or \subparagraph
\usepackage{etoolbox}
\makeatletter
\patchcmd\longtable{\par}{\if@noskipsec\mbox{}\fi\par}{}{}
\makeatother
% Allow footnotes in longtable head/foot
\IfFileExists{footnotehyper.sty}{\usepackage{footnotehyper}}{\usepackage{footnote}}
\makesavenoteenv{longtable}
\usepackage{graphicx,grffile}
\makeatletter
\def\maxwidth{\ifdim\Gin@nat@width>\linewidth\linewidth\else\Gin@nat@width\fi}
\def\maxheight{\ifdim\Gin@nat@height>\textheight\textheight\else\Gin@nat@height\fi}
\makeatother
% Scale images if necessary, so that they will not overflow the page
% margins by default, and it is still possible to overwrite the defaults
% using explicit options in \includegraphics[width, height, ...]{}
\setkeys{Gin}{width=\maxwidth,height=\maxheight,keepaspectratio}
% Set default figure placement to htbp
\makeatletter
\def\fps@figure{htbp}
\makeatother
\setlength{\emergencystretch}{3em} % prevent overfull lines
\providecommand{\tightlist}{%
  \setlength{\itemsep}{0pt}\setlength{\parskip}{0pt}}
\setcounter{secnumdepth}{5}
\usepackage{booktabs}
\usepackage{booktabs}
\usepackage{longtable}
\usepackage{array}
\usepackage{multirow}
\usepackage{wrapfig}
\usepackage{float}
\usepackage{colortbl}
\usepackage{pdflscape}
\usepackage{tabu}
\usepackage{threeparttable}
\usepackage{threeparttablex}
\usepackage[normalem]{ulem}
\usepackage{makecell}
\usepackage{xcolor}
\usepackage[]{natbib}
\bibliographystyle{apalike}

\title{Standar Pelayanan Minimal}
\author{Registrasi Mahasiswa Universitas Sultan Ageng Tirtayasa Tahun 2018}
\date{2020-04-12}

\begin{document}
\maketitle

{
\setcounter{tocdepth}{1}
\tableofcontents
}
\hypertarget{kata-pengantar}{%
\chapter*{KATA PENGANTAR}\label{kata-pengantar}}
\addcontentsline{toc}{chapter}{KATA PENGANTAR}

Penyempurnaan \textbf{Buku Standar Pelayanan Minimal} (\textbf{SPM}) Pelaksanaan Registrasi Mahasiswa tahun 2020 sesuai dengan perkembangan Ilmu dan Teknologi/IT dengan menggunakan e-administrasi (layanan administrasi akademik \emph{online}) Univeritas Sultan Ageng Tirtayasa, sebagai acuan dalam memberikan pelayanan dan informasi di Biro Akademik, Kemahasiswaan, dan Perencanaan (BAKP) terhadap mahasiswa Universitas Sultan Ageng Tirtayasa serta mempermudah mahasiswa dalam mengajukan permohonan registrasi (efesiensi waktu) sesuai kebutuhan mahasiswa.
Standar Pelayanan Minimal (SPM) Registrasi Mahasiswa ini merupakan penjabaran dari Pedoman Akademik, Program Kerja Universitas Sultan Ageng Tirtayasa tahun 2020, dan Program Kerja Biro Akademik, Kemahasiswaan, dan Perencanaan (BAKP) tahun 2020.

\textbf{Standar Pelayanan Minimal} (\textbf{SPM}) Registrasi Mahasiswan ini dasar pemikiran pelaksanaan administrasi \emph{online} (e-administrasi) agar dapat terpantau dan memiliki keseragaman kegiatan administrasi baik dalam proses maupun prosedur serta kebijakannya. Prinsip implementasi prosedur e-administrasi adalah :

\begin{enumerate}
\def\labelenumi{\arabic{enumi}.}
\tightlist
\item
  Memberikan pelayanan prima kepada mahasiswa,
\item
  Memberikan kepuasan terhadap layanan akademik terhadap mahasiswa,
\item
  Melakukan efesiensi penggunaan kertas dalam layanan akademik,
\item
  Penyergaman prosedur dan kebijakan dalam pelaksanaan layanan akademik,
\item
  Mendokumentasikan secara akurat proses layanan akademik mahasiswa. Pelaksanaan kegiatan registrasi mahasiswa di Universitas Sultan Ageng Tirtayasa sesuai dengan permasalahan pendidikan yang berkembang, maka Standar Pelayanan Minimal (SPM) Registrasi mahasiswa inipun terus dilakukan perbaikan dan penyempurnaan setiap waktu, sehingga sesuai dan selaras dengan kebutuhan dan dinamika perkembangan akademik.
\end{enumerate}

Kami berharap \textbf{Standar Pelayanan Minimal} (\textbf{SPM}) Registrasi Mahasiswa ini dapat berfungsi sebagai acuan dalam melaksanakan kegiatan registrasi mahasiswa baik pada tingkat Universitas, Fakultas, Jurusan atau Program Studi, Pimpinan, Mahasiswa, Dosen, dan Pegawai di lingkungan Universitas Sultan Ageng Tirtayasa.

Serang, April 2020

Kepala BAKP Untirta

Drs. Mochamad Ganiadi, M.M.

NIP. 19620422 199203 1 001

\hypertarget{intro}{%
\chapter{PENDAHULUAN}\label{intro}}

\hypertarget{latar-belakang}{%
\section{Latar Belakang}\label{latar-belakang}}

\textbf{Universitas Sultan Ageng Tirtayasa} dimulai dari Yayasan Pendidikan Tirtayasa (Yapenta) yang didirikan pada tanggal 1 Oktober 1980, Saat ini, Yapenta berubah menjadi Lapenta (Lembaga Pendidikan Tirtayasa). Yapenta berkedudukan dan bertempat di Kabupaten Serang. Pendirian Yapenta dikukuhkan dengan Akte Notaris Rosita Wibowo, SH, Nomor 1, tanggal 1 Oktober 1980. Kemudian dilakukan penyempurnaan dan dikukuhkan kembali dengan Akte Notaris Ny. R. Arie Soetardjo, Nomor 1, tanggal 3 Maret 1986.

Tujuan pendirian Yapenta adalah.

\begin{enumerate}
\def\labelenumi{\arabic{enumi}.}
\tightlist
\item
  Membantu usaha-usaha pemerintah dalam bidang pendidikan umum. Yaitu mulai dari taman kanak-kanak sampai dengan perguruan tinggi
\item
  Mendirikan sekolah-sekolah mulai dari taman kanak-kanak sampai dengan perguruan tinggi, termasuk juga sekolah-sekolah kejuruan.
\item
  Merencanakan dan mengusahakan sarana pendidikan, termasuk juga sarana olah raga.
\end{enumerate}

Kata Tirtayasa diambil dari nama pahlawan nasional yang berasal dari Banten, yaitu \textbf{Sultan Ageng Tirtayasa} (Kepres RI Nomor: 045/TK/1070). Nama asli Sultan Ageng Tirtayasa adalah Abul Fathi Abdul Fatah, pewaris kesultanan Banten keempat, yang dengan gigih menentang penjajahan Belanda dan berhasil membawa kejayaan dan keemasan Banten. Kata Tirtayasa sendiri berarti air mengalir (Sansekerta).

Pada awalnya Yayasan Pendidikan Tirtayasa (Yapenta) Banten menaungi Sekolah Tinggi Ilmu Hukum (STIH), Sekolah Tinggi Keguruan dan Ilmu Pendidikan (STKIP), dan Sekolah Tinggi Teknik (STT). STIH didirikan pada tanggal 1 Oktober 1980, sebagai embrio terbentuknya Universitas Tirtayasa (Untirta). Kemudian tanggal 1 Oktober 1980 disepakati sebagai tanggal kelahiran Untirta, sehingga upacara Dies Natalis Universitas Sultan Ageng Tirtayasa (Untirta) dilaksanakan tiap tanggal 1 Oktober.

Universitas Tirtayasa Banten merupakan penggabungan dari STIH, STT dan STKIP didasarkan pada SK Mendikbud RI Nomor:0596/0/1984, tanggal 28 Nopember 1984, ditingkatkan statusnya, sehingga menjadi Fakultas Hukum, Fakultas Teknik, dan Fakultas Ilmu Keguruan dan Pendidikan. Selanjutnya dengan SK Mendikbud RI Nomor: 0597/0/1984, tanggal 28 Nopember 1984, ketiga Fakultas tersebut ditetapkan sebagai status terdaftar.

Untirta berkembang dengan berdirinya Fakultas Pertanian dan Fakultas Ekonomi secara berturut-turut dengan SK Mendikbud RI Nomor: 0123/0/189, tanggal 8 Maret 1989, dan Nomor: 0331/0/1989, tanggal 30 Mei 1989, masing-masing dengan status terdaftar. Selanjutnya pada tanggal 13 Oktober 1999 keluar Keppres RI Nomor: 130/1999 tentang Persiapan Perguruan Tinggi Negeri Universitas Sultan Ageng Tirtayasa. Berdasarkan Keputusan Presiden RI Nomor: 32 tanggal 19 Maret 2001, Universitas Sultan Ageng Tirtayasa menjadi Perguruan Tinggi Negeri, maka Universitas Sultan Ageng Tirtayasa beralih dari naungan Yayasan Pendidikan Tirtayasa Banten (Yapenta) masuk kedalam lingkungan Departemen Pendidikan Nasional dan pengalihan aset serta pengelolaan sumber daya dari Yapenta kepada Pemerintah telah dilaksanakan. Tahun 2002.

Universitas Sultan Ageng Tirtayasa saat ini telah menyelenggarakan program pendidikan akademik dan program pendidikan vokasi. Program pendidikan akademik terdiri atas Program Pendidikan Sarjana (S1), Diploma (D3), sebanyaK enam fakultas dan satu program pendidikan Magister (Pascasarjana).

\hypertarget{fakultas-hukum}{%
\subsection{Fakultas Hukum}\label{fakultas-hukum}}

Memiliki satu jurusan/program studi strata satu (S1), yaitu : Ilmu Hukum kemudian berdasarkan keputusann Kemenristek Dikti nomor 257/M/KPT/2017 tentang nama program studi dan ditetapkan dengan SK Rektor Untirta Nomor 932/UN43/AK/SK/2017 menjadi Program Studi Hukum , dengan lima bidang studi yakni, Bidang Hukum Pidana, Bidang Hukum Perdata, Bidang Hukum Tata Negara, Bidang Hukum Administrasi Negara dan Bidang Hukum Internasional.

\hypertarget{fakultas-keguruan-dan-ilmu-pendidikan}{%
\subsection{Fakultas Keguruan dan Ilmu Pendidikan}\label{fakultas-keguruan-dan-ilmu-pendidikan}}

Memiliki 18 jurusan program strata satu (S1), mengalami penyesuaian berdasarkan Kemenristek Dikti nomor 257/M/KPT/2017 yaitu :

\begin{enumerate}
\def\labelenumi{\arabic{enumi}.}
\tightlist
\item
  Pendidikan Luar Sekolah (PLS) menjadi Pendidikan Nonformal,\\
\item
  Pendidikan Guru Sekolah Dasar (PGSD),
\item
  Pendidan Guru Pendidikan Anak usia Dini (PGPAUD),
\item
  Pendidikan Luar Biasa menjadi Pendidikan Khusus,
\item
  Bimbingan dan Konseling,
\item
  Pendidikan Bahasa dan Sastra Indonesia (Diksastrasia) menjadi Pendidikan Bahasa Indonesia,
\item
  Pendidikan Bahasa Inggris,
\item
  Pendidikan Seni Drama Tari dan Musik (Sendratasik) menjadi Pendidikan Seni Pertunjukan,
\item
  Pendidikan Matematika,
\item
  Pendidikan Biologi,
\item
  Pendidikan Fisika,
\item
  Pendidikan Kimia,
\item
  Pendidikan IPA,
\item
  Pendidikan Sejarah,
\item
  Pendidikan Pancasila dan Kewarganegaraan,
\item
  Pendidikan Sosiologi,
\item
  Pendidikan Teknik Elektro menjadi Pendidikan Vokasional Teknik Elektro,
\item
  Pendidikan Teknik Mesin menjadi Pendidikan Vokasional Teknik Mesin,
\end{enumerate}

\hypertarget{fakultas-teknik}{%
\subsection{Fakultas Teknik}\label{fakultas-teknik}}

Memiliki enam jurusan program strata satu (S1), yaitu:

\begin{enumerate}
\def\labelenumi{\arabic{enumi}.}
\tightlist
\item
  Teknik Mesin,\\
\item
  Teknik Elektro,\\
\item
  Teknik Industri,\\
\item
  Teknik Metalurgi,
\item
  Teknik Kimia,
\item
  Teknik Sipil.
\end{enumerate}

\hypertarget{fakultas-pertanian}{%
\subsection{Fakultas Pertanian}\label{fakultas-pertanian}}

Memiliki tiga jurusan program strata satu (S1), yaitu:

\begin{enumerate}
\def\labelenumi{\arabic{enumi}.}
\tightlist
\item
  Agribisnis,\\
\item
  Agroekoteknologi,\\
\item
  Perikanan,
\item
  Teknologi Pangan.
\end{enumerate}

\hypertarget{fakultas-ekonomi-dan-bisnis}{%
\subsection{Fakultas Ekonomi dan Bisnis}\label{fakultas-ekonomi-dan-bisnis}}

Memiliki empat jurusan program strata satu (S1) dan empat program studi Diploma (D3), mengalami penyesuaian berdasarkan Kemenristek Dikti nomor 257/M/KPT/2017, yaitu:

\begin{enumerate}
\def\labelenumi{\arabic{enumi}.}
\tightlist
\item
  Manajemen,
\item
  Akuntansi,
\item
  Ekonomi Studi Pembangunan,
\item
  Ekonomi Islam menjadi Ekonomi Syariah
\item
  Akuntansi (D3),\\
\item
  Marketing (D3) menjadi Manajemen Pemasaran,\\
\item
  Perpajakan (D3),
\item
  Keuangan dan Perbankan (D3) menjadi Perbankan dan Keuangan.
\end{enumerate}

\hypertarget{fakultas-ilmu-sosial-dan-ilmu-politik}{%
\subsection{Fakultas Ilmu Sosial dan Ilmu Politik}\label{fakultas-ilmu-sosial-dan-ilmu-politik}}

Memiliki tiga jurusan program strata satu (S1), yaitu:

\begin{enumerate}
\def\labelenumi{\arabic{enumi}.}
\tightlist
\item
  Ilmu Administrasi Negar menjadi Administrasi Publik,\\
\item
  Ilmu Komunikasi,
\item
  Ilmu Pemerintahan.
\end{enumerate}

\hypertarget{fakultas-kedokteran}{%
\subsection{Fakultas Kedokteran}\label{fakultas-kedokteran}}

Memiliki tiga jurusan program strata satu (S1) dan satu program studi Diploma (D3), yaitu:

\begin{enumerate}
\def\labelenumi{\arabic{enumi}.}
\tightlist
\item
  Kedokteran
\item
  Gizi
\item
  Ilmu Keolahragaan
\item
  Keperawatan (D3)
\end{enumerate}

\hypertarget{pascasarjana}{%
\subsection{Pascasarjana}\label{pascasarjana}}

Memiliki program Magister strata dua (S2) dengan enam program studi, yaitu:

\begin{enumerate}
\def\labelenumi{\arabic{enumi}.}
\tightlist
\item
  Pendidikan Bahasa Indonesia,
\item
  Teknologi Pendidikan,
\item
  Hukum,
\item
  Magister Akuntansi,
\item
  Magister Manajemen,
\item
  Magister Administrasi Publik.
\item
  Pendidikan Bahasa Inggris,
\item
  Pendidikan Matematika,
\item
  Ilmu Pertanian.
\item
  Teknik Kimia
\item
  Ilmu Komunikasi
\end{enumerate}

\hypertarget{visi-misi-tujuan-sasaran-dan-strategi-universitas-sultan-ageng-tirtayasa}{%
\section{Visi, Misi, Tujuan, Sasaran, dan Strategi Universitas Sultan Ageng Tirtayasa}\label{visi-misi-tujuan-sasaran-dan-strategi-universitas-sultan-ageng-tirtayasa}}

\hypertarget{visi-universitas-sultan-ageng-tirtayasa}{%
\subsection{Visi Universitas Sultan Ageng Tirtayasa}\label{visi-universitas-sultan-ageng-tirtayasa}}

\begin{quote}
``Terwujudnya UNTIRTA Sebagai Integrated, Smart, and Green (It'S Green) University yang UNGGUL, BERKARAKTER, DAN BERDAYA SAING, di Kawasan ASEAN tahun 2030''
\end{quote}

Berdasarkan visi tersebut di atas dapat dijelaskan antara lain sebagai berikut:

\begin{enumerate}
\def\labelenumi{\arabic{enumi}.}
\item
  \textbf{Maju}. Mengandung pengertian terwujudnya kondisi Untirta yang mengalami pertumbuhan, peningkatan dan perubahan secara berkelanjutan dalam penyelenggaraan pendidikan dan pembelajaran, penelitian, dan pengabdian kepada masyarakat, daya dukung sumber daya dan manajemen serta kerjasama kemitraan.
\item
  \textbf{Bermutu}. Mengandung pengertian tercapainya kualitas layanan yang memberikan kepuasan kepada pelanggan, lulusan Universitas Sultan Ageng Tirtayasa yang menguasai Iptek (hard skill), mampu berkolaborasi dan membangun jejaring (networking) berkomunkasi (soft skill) menuju kemajuan bangsa, peradaban dan kesejahteraan umat manusia.
\item
  \textbf{Berdaya Saing}. Mengandung pengertian terwujudnya suatu dorongan pada diri pendidik (dosen, tenaga kependidikan, dan lulusan untuk memenangkan suatu persaingan (kompetisi), lebih berprestasi, memiliki keunggulan komparatif dan keunggulan kompetitif, berupaya lebih baik dari yang lain, tahan menghadapi berbagai kondisi, hambatan dan tantangan serta mampu beradaptasi dengan lingkungan.
\item
  \textbf{Berkarakter}. Mengandung arti tercapainya tenaga pendidik dan kependidikan serta lulusan universitas Sultan Ageng Tirtayasa yang menguasai Iptek dengan menjunjung tinggi kejujuran, amanah, berwibawa, adil, religius, dan akuntabel.
\item
  \textbf{Kebersamaan}. Dalam mewujudkan misi Untirta perlu terbangun komunikasi kerja di Universitas Sultan Ageng Tirtayasa lebih mengutamakan semangat gotong royong, kolegial, saling pengertian, saling menghargai dan saling menghormati, sebagai sebuah Tim kerja yang menjunjung tinggi solidaritas dan soliditas. Hal ini meniscayakan seluruh komponen Untirta mulai dari level teratas sampai dengan level terbawah bersama-sama berkomitmen memberikan.
\end{enumerate}

\hypertarget{misi-universitas-sultan-ageng-tirtayasa}{%
\subsection{Misi Universitas Sultan Ageng Tirtayasa}\label{misi-universitas-sultan-ageng-tirtayasa}}

Untuk mencapai Visi di atas, Universitas Sultan Ageng Tirtayasa menetapkan misi sebagai berikut:

\begin{enumerate}
\def\labelenumi{\arabic{enumi}.}
\tightlist
\item
  Meningkatkan kualitas, relevansi dan daya saing pendidikan serta lulusan yang unggul, berkarakter, serta berdaya saing di kawasan ASEAN.
\item
  Meningkatkan kualitas dan kuantitas penelitian dan pengabdian kepada masyarakat yang inovatif berbasis kebutuhan nyata sesuai perkembangan zaman.
\item
  Meningkatkan daya dukung tatakelola perguruan tinggi yang baik sebagai implementasi dari Integrated Smart and Green (It'S Green) University.
\end{enumerate}

\hypertarget{tujuan-universitas-sultan-ageng-tirtayasa}{%
\subsection{Tujuan Universitas Sultan Ageng Tirtayasa}\label{tujuan-universitas-sultan-ageng-tirtayasa}}

\begin{enumerate}
\def\labelenumi{\arabic{enumi}.}
\tightlist
\item
  Menghasilkan lulusan yang berkualitas, terdidik, terlatih, berdaya saing, dan berkarakter sesuai kebutuhan \emph{stakeholders}.
\item
  Menghasilkan penelitan dan pengabdian kepada masyarakat yang inovatif berbasis kebutuhan nyata serta berorientasi pada pemanfaatan oleh dunia industri, pembangunan daerah, dan masyarakat.
\item
  Menghasilkan daya dukung tatakelola yang efektif, efisien, transparan, dan akuntabel dalam mengembangkan tri dharma perguruan tinggi.
\end{enumerate}

\hypertarget{sasaran}{%
\subsection{Sasaran}\label{sasaran}}

Dalam rangka mencapai tujuan tersebut, perlu ditetapkan sasaran sebagai berikut:

\begin{enumerate}
\def\labelenumi{\arabic{enumi}.}
\tightlist
\item
  Meningkatnya kualitas, kuantitas, relevansi, dan daya saing lulusan.
\item
  Meningkatnya kualitas dan kuantitas Program Studi sesuai Kebutuhan \emph{Stakeholders}.
\item
  Meningkatnya kualitas dan kuantitas penelitian dan pengabdian kepada masyarakat.
\item
  Tersedianya daya dukung SDM sesuai Standar Kompetensi.
\item
  Meningkatnya daya dukung administrasi akademik dan non akademik.
\item
  Tersedianya daya dukung sarana prasarana yang memadai.
\item
  Kerjasama kemitraaan strategis nasional dan internasional.
\end{enumerate}

\hypertarget{strategi}{%
\subsection{Strategi}\label{strategi}}

\begin{enumerate}
\def\labelenumi{\arabic{enumi}.}
\tightlist
\item
  Penguatan Kualitas Layanan Pendidikan dan Organisasi Kemahasiswaan.
\item
  Penguatan Program Studi (akademik, vokasi, dan profesi) memenuhi standar Mutu Pendidikan Tinggi.
\item
  Penguatan Penelitian dan Pengabdian kepada Masyarakat yang inovatif berbasis kebutuhan industri, pembangunan daerah, dan masyarakat.
\item
  Penguatan SDM dosen dan tenaga kependidikan sesuai kompetensi.
\item
  Penguatan layanan administasi akademik dan nonakademik.
\item
  Penguatan Sarana dan prasarana perguruan tinggi.
\item
  Penguatan Kerjasama dan kemitraan.
\end{enumerate}

\hypertarget{pendidikan-tinggi}{%
\section{Pendidikan Tinggi}\label{pendidikan-tinggi}}

Program pendidikan tinggi yang diselenggarakan oleh Universitas Sultan Ageng Tirtayasa: (1) Program Pendidikan Akademik, (2) Program Pendidikan Vokasi. Adapun pengertiannya adalah sebagai berikut:

\hypertarget{program-pendidikan-akademik}{%
\subsection{Program Pendidikan Akademik}\label{program-pendidikan-akademik}}

Bertujuan menyiapkan peserta didik untuk menjadi anggota masyarakat yang memiliki kemampuan akademik dalam menerapkan, mengembangkan, dan/atau memperkaya khasanah ilmu pengetahuan, teknologi, dan/atau kesenian, serta menyebarluaskan dan mengupayakan penggunaannya untuk meningkatkan taraf kehidupan masyarakat dan memperkaya kebudayaan nasional. Program pendidikan akademik di Untirta terdiri atas Program Sarjana (S1) dan Program Pascasarjana (S2).

\hypertarget{tujuan-pendidikan-program-magister}{%
\subsubsection{Tujuan Pendidikan Program Magister}\label{tujuan-pendidikan-program-magister}}

Program Magister diarahkan pada hasil lulusan yang memiliki kualifikasi sebagai berikut:

\begin{enumerate}
\def\labelenumi{\arabic{enumi}.}
\tightlist
\item
  Mampu menguasai perkembangan ilmu pengetahuan, teknologi,dan/atau seni dalam bidangnya dengan cara menguasai dan memahami teori-teori yang mutakhir, pendekatan, metode, dan kaidah-kaidah ilmiah disertai penerapannya.
\item
  Mampu memecahkan permasalahan di bidang keahliannya melalui kegatan penelitian dan pengembangan berdasarkan kaidah ilmiah.
\item
  Mampu mengembangkan kinerja profesionalnya yang ditunjukkan dengan ketajaman analisis permasalahan dan kepaduan pemecahan masalah.
\item
  Mampu berkomunikasi efektif termasuk berbahasa internasional.
\end{enumerate}

\hypertarget{tujuan-program-sarjana}{%
\subsubsection{Tujuan Program Sarjana}\label{tujuan-program-sarjana}}

Program Sarjana diarahkan pada hasil lulusan yang memiliki kualifikasi sebagai berikut:

\begin{enumerate}
\def\labelenumi{\arabic{enumi}.}
\tightlist
\item
  Menguasai dasar-dasar ilmiah dan keterampilan dalam bidang keahlian tertentu sehingga mampu menemukan, memahami, menjelaskan, dan merumuskan cara penyelesaian masalah yang ada di dalam kawasan keahliannya.
\item
  Mampu menerapkan ilmu pengetahuan dan keterampilan yang dimilikinya sesuai dengan bidang keahliannya dalam kegiatan produktif dan pelayanan kepada masyarakat dengan sikap dan perilaku yang sesuai dengan tata kehidupan bermasyarakat.
\item
  Mampu bersikap dan berperilaku dalam membawa diri berkarya di bidang keahliannya maupun dalam kehidupan bersama dalam masyarakat.
\item
  Mampu mengikuti perkembangan ilmu pengetahuan, teknologi, dan/atau kesenian yang merupakan keahliannya.
\item
  Mampu bersaing dan beradaptasi dalam lingkungan persaingan global.
\end{enumerate}

\hypertarget{program-pendidikan-vokasi}{%
\subsection{Program Pendidikan Vokasi}\label{program-pendidikan-vokasi}}

Bertujuan menyiapkan peserta didik menjadi anggota masyarakat yang memiliki pengetahuan dan keterampilan di bidang kerja serta memiliki tanggungjawab profesional terhadap pekerjaannya, serta mampu melaksanakan pengawasan dan bimbingan atas dasar keterampilan manajerial yang dimilikinya. Program pendidikan vokasional yang diselenggarakan adalah Program Diploma III.

\hypertarget{tujuan-pendidikan-program-vokasi}{%
\subsubsection{Tujuan Pendidikan Program Vokasi}\label{tujuan-pendidikan-program-vokasi}}

Program vokasi diarahkan untuk menghasilkan tenaga ahli madya yang memiliki kualifikasi sebagai berikut:

\begin{enumerate}
\def\labelenumi{\arabic{enumi}.}
\tightlist
\item
  Mampu menguasai perkembangan ilmu pengetahuan, teknologi, dan/atau seni dalam bidangnya kearah kegiatan yang produktif.
\item
  Terampil dan profesional kearah pemecahan masalah serta pelayanan langsung kepada masyarakat sesuai dengan bidang keahliannya.
\item
  Memiliki integritas kepribadian yang tinggi serta berjiwa entrepreneurial.
\end{enumerate}

\textbf{Program Studi, Kode Fakultas, dan Kode Jurusan/Program Studi Universitas Sultan Ageng Tirtayasa}

\begin{enumerate}
\def\labelenumi{\arabic{enumi}.}
\tightlist
\item
  Program Pascasarjana (S2)
\end{enumerate}

\begin{longtable}{r|l|r|r}
\hline
\multicolumn{2}{c|}{ } & \multicolumn{2}{c}{Kode} \\
\cline{3-4}
No & Program Studi & Fakultas & Program Studi\\
\hline
\endfirsthead
\multicolumn{4}{@{}l}{\textit{(continued)}}\\
\hline
No & Program Studi & Fakultas & Program Studi\\
\hline
\endhead
1 & Pendidikan Bahasa Indonesia & 77 & 71\\
\hline
2 & Teknologi Pendidikan & 77 & 72\\
\hline
3 & Hukum & 77 & 73\\
\hline
4 & Magister Akuntansi & 77 & 74\\
\hline
5 & Magister Administrasi Publik & 77 & 75\\
\hline
6 & Magister Manajemen & 77 & 76\\
\hline
7 & Pendidikan Bahasa Inggris & 77 & 77\\
\hline
8 & Pendidikan Matematika & 77 & 78\\
\hline
9 & Ilmu Pertanian & 77 & 79\\
\hline
10 & Teknik Kimia & 77 & 80\\
\hline
11 & Ilmu Komunikasi & 77 & 81\\
\hline
12 & Pendidikan (S3) & 77 & 82\\
\hline
\end{longtable}

\begin{enumerate}
\def\labelenumi{\arabic{enumi}.}
\setcounter{enumi}{1}
\tightlist
\item
  Program Sarjana (S1)
\end{enumerate}

\begin{longtable}{l|r|r}
\hline
\multicolumn{1}{c|}{ } & \multicolumn{2}{c}{Kode} \\
\cline{2-3}
Fakultas/Program Studi & Fakultas & Program Studi\\
\hline
\endfirsthead
\multicolumn{3}{@{}l}{\textit{(continued)}}\\
\hline
Fakultas/Program Studi & Fakultas & Program Studi\\
\hline
\endhead
\hspace{1em}Hukum & 11 & 11\\
\hline
\multicolumn{3}{l}{\textbf{Keguruan dan Ilmu Pendidikan}}\\
\hline
\hspace{1em}Pendidikan Nonformal & 22 & 21\\
\hline
\hspace{1em}Pendidikan Bahasa Indonesia & 22 & 22\\
\hline
\hspace{1em}Pendidikan Bahasa Inggris & 22 & 23\\
\hline
\hspace{1em}Pendidikan Biologi & 22 & 24\\
\hline
\hspace{1em}Pendidikan Matematika & 22 & 25\\
\hline
\hspace{1em}Pendidikan Guru Sekolah Dasar & 22 & 27\\
\hline
\hspace{1em}Pendidikan Guru Pendidikan Anak Usia Dini & 22 & 28\\
\hline
\hspace{1em}Pendidikan Fisika & 22 & 80\\
\hline
\hspace{1em}Pendidikan IPA & 22 & 81\\
\hline
\hspace{1em}Pendidikan Kimia & 22 & 82\\
\hline
\hspace{1em}Pendidikan (Vokasional) Teknik Elektro & 22 & 83\\
\hline
\hspace{1em}Pendidikan (Vokasional) Teknik Mesin & 22 & 84\\
\hline
\hspace{1em}Bimbingan dan Konseling & 22 & 85\\
\hline
\hspace{1em}Pendidikan Pancasila dan Kewarganegraan & 22 & 86\\
\hline
\hspace{1em}Pendidikan Khusus & 22 & 87\\
\hline
\hspace{1em}Pendidikan Sejarah & 22 & 88\\
\hline
\hspace{1em}Pendidikan Seni Pertunjukan & 22 & 89\\
\hline
\hspace{1em}Pendidikan Sosiologi & 22 & 90\\
\hline
\multicolumn{3}{l}{\textbf{Teknik}}\\
\hline
\hspace{1em}Teknik Mesin & 33 & 31\\
\hline
\hspace{1em}Teknik Elektro & 33 & 32\\
\hline
\hspace{1em}Teknik Industri & 33 & 33\\
\hline
\hspace{1em}Teknik Metalurgi & 33 & 34\\
\hline
\hspace{1em}Teknik Kimia & 33 & 35\\
\hline
\hspace{1em}Teknik Sipil & 33 & 36\\
\hline
\multicolumn{3}{l}{\textbf{Pertanian}}\\
\hline
\hspace{1em}Agribisnis & 44 & 41\\
\hline
\hspace{1em}Agroekoteknologi & 44 & 42\\
\hline
\hspace{1em}Perikanan & 44 & 43\\
\hline
\hspace{1em}Teknologi Pangan & 44 & 44\\
\hline
\multicolumn{3}{l}{\textbf{Ekonomi dan Bisnis}}\\
\hline
\hspace{1em}Manajemen & 55 & 51\\
\hline
\hspace{1em}Akuntansi & 55 & 52\\
\hline
\hspace{1em}Ekonomi Pembangunan & 55 & 53\\
\hline
\hspace{1em}Ekonomi Syariah & 55 & 54\\
\hline
\multicolumn{3}{l}{\textbf{Ilmu Sosial dan Ilmu Politik}}\\
\hline
\hspace{1em}Administrasi Publik & 66 & 61\\
\hline
\hspace{1em}Ilmu Komunikasi & 66 & 62\\
\hline
\hspace{1em}Ilmu Pemerintahan & 66 & 70\\
\hline
\multicolumn{3}{l}{\textbf{Kedokteran}}\\
\hline
\hspace{1em}Kedokteran & 88 & 81\\
\hline
\hspace{1em}Gizi & 88 & 82\\
\hline
\hspace{1em}Ilmu Keolahragaan & 88 & 83\\
\hline
\end{longtable}

\begin{enumerate}
\def\labelenumi{\arabic{enumi}.}
\setcounter{enumi}{2}
\tightlist
\item
  Program Diploma (D3)
\end{enumerate}

\hypertarget{registrasi-mahasiswa-baru}{%
\chapter{REGISTRASI MAHASISWA BARU}\label{registrasi-mahasiswa-baru}}

\hypertarget{ketentuan-mahasiswa-baru}{%
\section{Ketentuan Mahasiswa Baru}\label{ketentuan-mahasiswa-baru}}

\hypertarget{waktu-registrasi}{%
\section{Waktu Registrasi}\label{waktu-registrasi}}

\hypertarget{prosedur-persyaratan-mahasiswa-baru}{%
\section{Prosedur Persyaratan Mahasiswa Baru}\label{prosedur-persyaratan-mahasiswa-baru}}

\hypertarget{biro-akademik-kemahasiswaan-dan-perencanaan-bakp}{%
\section{Biro Akademik, Kemahasiswaan, dan Perencanaan (BAKP)}\label{biro-akademik-kemahasiswaan-dan-perencanaan-bakp}}

\hypertarget{biro-umum-keuangan-dan-kepegawaian}{%
\section{Biro Umum, Keuangan, dan Kepegawaian}\label{biro-umum-keuangan-dan-kepegawaian}}

\hypertarget{bank-bni-46}{%
\section{Bank (BNI 46)}\label{bank-bni-46}}

\hypertarget{subbagian-registrasi-dan-statistik---bakp}{%
\section{Subbagian Registrasi dan Statistik - BAKP}\label{subbagian-registrasi-dan-statistik---bakp}}

\hypertarget{pusat-data-dan-informasi-pusdainfo}{%
\section{Pusat Data dan Informasi (PUSDAINFO)}\label{pusat-data-dan-informasi-pusdainfo}}

\hypertarget{prosedur-kontrak-mata-kuliah}{%
\section{Prosedur Kontrak Mata Kuliah}\label{prosedur-kontrak-mata-kuliah}}

\hypertarget{petugas-registrasi}{%
\section{Petugas Registrasi}\label{petugas-registrasi}}

\hypertarget{registrasi-mahasiswa-lama}{%
\chapter{REGISTRASI MAHASISWA LAMA}\label{registrasi-mahasiswa-lama}}

\hypertarget{ketentuan-mahasiswa-lama}{%
\section{Ketentuan Mahasiswa Lama}\label{ketentuan-mahasiswa-lama}}

\hypertarget{waktu-registrasi-1}{%
\section{Waktu Registrasi}\label{waktu-registrasi-1}}

\hypertarget{prosedur-kontrak-mata-kuliah-1}{%
\section{Prosedur Kontrak Mata Kuliah}\label{prosedur-kontrak-mata-kuliah-1}}

\hypertarget{petugas-registrasi-1}{%
\section{Petugas Registrasi}\label{petugas-registrasi-1}}

\hypertarget{pengajuan-ijin-cuti-kuliah}{%
\chapter{PENGAJUAN IJIN CUTI KULIAH}\label{pengajuan-ijin-cuti-kuliah}}

\hypertarget{ketentuan-pengajuan-cuti-kuliah}{%
\section{Ketentuan Pengajuan Cuti Kuliah}\label{ketentuan-pengajuan-cuti-kuliah}}

\hypertarget{waktu-registrasi-2}{%
\section{Waktu Registrasi}\label{waktu-registrasi-2}}

\hypertarget{prosedur-pengajuan-cuti-kuliah}{%
\section{Prosedur Pengajuan Cuti Kuliah}\label{prosedur-pengajuan-cuti-kuliah}}

\hypertarget{petugas-registrasi-2}{%
\section{Petugas Registrasi}\label{petugas-registrasi-2}}

\hypertarget{pengajuan-aktif-kuliah-kembali}{%
\chapter{PENGAJUAN AKTIF KULIAH KEMBALI}\label{pengajuan-aktif-kuliah-kembali}}

\hypertarget{ketentuan-pengajuan-aktif-kuliah-kembali}{%
\section{Ketentuan Pengajuan Aktif Kuliah Kembali}\label{ketentuan-pengajuan-aktif-kuliah-kembali}}

\hypertarget{waktu-registrasi-3}{%
\section{Waktu Registrasi}\label{waktu-registrasi-3}}

\hypertarget{prosedur-pengajuan-aktif-kuliah-kembali}{%
\section{Prosedur Pengajuan Aktif Kuliah Kembali}\label{prosedur-pengajuan-aktif-kuliah-kembali}}

\hypertarget{prosedur-kontrak-mata-kuliah-2}{%
\section{Prosedur Kontrak Mata Kuliah}\label{prosedur-kontrak-mata-kuliah-2}}

\hypertarget{petugas-registrasi-3}{%
\section{Petugas Registrasi}\label{petugas-registrasi-3}}

\hypertarget{pengajuan-pindah-program-studi}{%
\chapter{PENGAJUAN PINDAH PROGRAM STUDI}\label{pengajuan-pindah-program-studi}}

\hypertarget{ketentuan-umum-pengajuan-pindah-program-studi}{%
\section{Ketentuan Umum Pengajuan Pindah Program Studi}\label{ketentuan-umum-pengajuan-pindah-program-studi}}

\hypertarget{waktu-registrasi-4}{%
\section{Waktu Registrasi}\label{waktu-registrasi-4}}

\hypertarget{prosedur-pengajuan-pindah-program-studi}{%
\section{Prosedur Pengajuan Pindah Program Studi}\label{prosedur-pengajuan-pindah-program-studi}}

\hypertarget{prosedur-kontrak-mata-kuliah-3}{%
\section{Prosedur Kontrak Mata Kuliah}\label{prosedur-kontrak-mata-kuliah-3}}

\hypertarget{petugas-registrasi-4}{%
\section{Petugas Registrasi}\label{petugas-registrasi-4}}

\hypertarget{pengajuan-pindah-kuliah-ke-pt-lain}{%
\chapter{PENGAJUAN PINDAH KULIAH KE PT LAIN}\label{pengajuan-pindah-kuliah-ke-pt-lain}}

\hypertarget{ketentuan-umum-pengajuan-pindah-kuliah-ke-pt-lain}{%
\section{Ketentuan Umum Pengajuan Pindah Kuliah ke PT Lain}\label{ketentuan-umum-pengajuan-pindah-kuliah-ke-pt-lain}}

\hypertarget{waktu-registrasi-5}{%
\section{Waktu Registrasi}\label{waktu-registrasi-5}}

\hypertarget{prosedur-pengajuan-pindah-kuliah-keluar-dari-untirta}{%
\section{Prosedur Pengajuan Pindah Kuliah (Keluar dari Untirta)}\label{prosedur-pengajuan-pindah-kuliah-keluar-dari-untirta}}

\hypertarget{petugas-registrasi-5}{%
\section{Petugas Registrasi}\label{petugas-registrasi-5}}

\hypertarget{permohonan-pernyataan-masih-kuliah}{%
\chapter{PERMOHONAN PERNYATAAN MASIH KULIAH}\label{permohonan-pernyataan-masih-kuliah}}

\hypertarget{ketentuan-pengajuan-surat-pernyataan-masih-kuliah}{%
\section{Ketentuan Pengajuan Surat Pernyataan Masih Kuliah}\label{ketentuan-pengajuan-surat-pernyataan-masih-kuliah}}

\hypertarget{waktu-registrasi-6}{%
\section{Waktu Registrasi}\label{waktu-registrasi-6}}

\hypertarget{prosedur-pengajuan-surat-pernyataan-masih-kuliah}{%
\section{Prosedur Pengajuan Surat Pernyataan Masih Kuliah}\label{prosedur-pengajuan-surat-pernyataan-masih-kuliah}}

\hypertarget{alih-jenjang}{%
\chapter{ALIH JENJANG}\label{alih-jenjang}}

\hypertarget{dari-luar-ke-untirta}{%
\section{Dari Luar ke Untirta}\label{dari-luar-ke-untirta}}

\hypertarget{d3-feb-ke-s1}{%
\section{D3 FEB ke S1}\label{d3-feb-ke-s1}}

  \bibliography{book.bib,packages.bib}

\end{document}
